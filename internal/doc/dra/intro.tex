\Chapter{Introduction\label{sec:intro}}

\Section{Overview\label{sec:overview}}

This document is a user manual for \about{dra}, an interpreter for tabled
logic programming with coinduction.  The manual has been written for users of
various versions of Unix: if you are running another system, various small
details (such as the way to enter the end-of-file character) may be
different.%
\footnote{
  All comments, queries and suggestions about this manual or the software
  will be most welcome: my e-mail address is
  \prog{feliks.kluzniak@utdallas.edu}.  Please note that I will not be able
  to provide assistance with Windows.}

The interpreter implements ``top-down tabled programming'' via so called
``Dynamic Reordering of Alternatives'' \cite{guo-gupta-dra}.  It also
supports ``co-logic programming'', i.e., logic programs that contain
coinductive predicates \cite{coinductive}, \cite{co-LP}.


Apart from support for coinduction, there are two significant changes with
respect to the original description \cite{guo-gupta-dra}:
\begin{enumerate}

\item
  A tabled goal will never produce the same answer twice.\footnote{
  Please note that in this document \about{goal} means an instance of a
  procedure call.}

  More specifically: two answers will never be variants of each
  other.

\item
  By default, new results for a tabled goal will be produced before old
  answers.  The user can reverse the order by issuing the directive\\
  \ind\prog{:-~old\_first.}

  A ``new result for a tabled goal'' is a result that has not yet been
  tabled for this goal.  (More precisely: the table does not contain a
  variant of this result associated with a variant of this goal.)

  The default behaviour is intended to help computations converge more
  quickly.  The user is given an option to change it, because some predicates
  may produce a very large (even infinite) set of answers on backtracking,
  and the application might not require those answers.
\end{enumerate}


%-------------------------------------------------------------------------------
\Section{Tabling\label{sec:tabling}}

When you declare a predicate as tabled\index{tabled predicate}, there will be
two consequences:
\begin{enumerate}

\item Every time a goal with this predicate symbol (i.e., an invocation of
  this predicate) succeeds, the result is stored in a table.  Such a tabled
  result is commonly referred to as an ``answer''.\index{answer}

  Once all the possible results have been so stored, subsequent invocations
  can draw upon the tabled answers, instead of recomputing them all over
  again. This is all done in a way that does not affect the semantics of the
  program, except perhaps for the order in which results are reported and the
  number of repetitions of the same result. (Of course, if your program is
  not sufficiently close to pure Prolog, e.g., if it relies on and/or
  produces side-effects, then all bets are off.)

  This may not always be as advantageous as one might think, for two reasons:
  \begin{enumerate}
  \item If the number of results is large, and goals are seldom repeated, the
    main effect might be a significantly increased demand for memory, which
    is not offset by shorter processing time.

  \item It so happens that in order to preserve the semantics of the program
    the results must be associated with the goal that produced them, and are
    accessible only to variant goals.  Other goals that invoke the same
    predicate will require recomputation, and the computed results will
    increase the size of the table, even if they are the same as those that
    have already been stored for another goal.
  \end{enumerate}

\item When a goal that invokes a tabled predicate is found to be a variant of
  one of its ancestors on the recursion stack, the goal is not
  expanded.\footnote{
    Instead, it succeeds with results that have already been put into the
    table (or fails if there are none).  Things are arranged so that all the
    results that can be computed will eventually wind up in the table.}
  This allows you to write your predicates in a more declarative fashion,
  without worrying, e.g., about the danger of left recursion.
\end{enumerate}
A more extensive discussion of tabling is well outside the scope of this user
manual.



%-------------------------------------------------------------------------------
\Section{Coinduction\label{sec:coindution}}

When you declare a predicate as coinductive\index{coinductive predicate},
there is only one effect:
\begin{itemize}
  \item
    If a goal that invokes a coinductive predicate is found to be unifiable
    with an ancestor on the recursion stack, the goal is unified with the
    ancestor and succeeds.  Upon backtracking it is unified---one by
    one---with other such ancestors\footnote{
      In \about{dra} this is done in reverse chronological order, i.e.,
      the more immediate ancestors are used first.
    }, and expanded in the normal way (i.e., by
    using clauses) only when all the unifiable ancestors have been taked
    advantage of in this manner.
\end{itemize}
This gives the predicate a radically new meaning, which is often the
appropriate one for operations on ``infinite'' terms (represented by cyclic
terms).  Again, further discussion of the concept is outside the scope of
this document.
