\documentclass[12pt,final]{report}
\usepackage{fancyvrb}
\usepackage{float}
\usepackage{varioref}
\usepackage[nottoc]{tocbibind} % Doesn't play well with package "index" :-(
\usepackage{makeidx}
%\usepackage{index}  % prevents expansion in index entries, needed for progidx
                     % Makes hyperref for index go awry, so decided to use
                     % explicit protect instead: turned out not to be neeeded

%\proofmodetrue      %  <<<< (index) suppress for the final copy <<<<<<


\usepackage{parskip}[2001/04/09]
%\setlength{\parindent}{0pt}
%\setlength{\parskip}{2ex}

\usepackage[colorlinks]{hyperref}

\makeindex

%----------------------------------------------------------------------
%  MACROS

% Facilitate transition between article and book:
\newcommand*{\Chapter}[1]{\chapter{#1}}
\newcommand*{\Section}[1]{\section{#1}}
\newcommand*{\Subsection}[1]{\subsection{#1}}
%\newcommand*{\Chapter}[1]{\section{#1}}
%\newcommand*{\Section}[1]{\subsection{#1}}
%\newcommand*{\Subsection}[1]{\subsubsection{#1}}

% Three asterisks break the text up where even a subsection is not merited:
% use this as a separate paragraph:
\newcommand*{\Breakup}{\[\ast\ast\ast\]}

% Definition of a new concept:
\newcommand*{\Defconcept}[1]{\emph{#1}\index{#1}}
% Ditto if the word is different from the index entry:
\newcommand*{\Defconcepti}[2]{\emph{#1}\index{#2}}

% A short form of a tiny \marginpar:
\newcommand*{\mpar}[1]{\marginpar{\tiny#1}}

% A word in the text that should be indexed:
\newcommand*{\Index}[1]{#1\index{#1}}

% A word that is the object of discussion:
\newcommand*{\about}[1]{\emph{#1}}

% A small piece of a concrete program etc.:
\newcommand*{\prog}[1]{\texttt{#1}}

% A ``pattern'', e.g., a generic variable within a concrete call:
\newcommand*{\patt}[1]{\textit{#1}}

% A reference to an entity in a program:
\newcommand*{\progfrag}[1]{\about{#1}}

% An example of a term in the text:
\newcommand*{\term}[1]{\progfrag{#1}}

% A predicate specification:
\newcommand*{\pred}[1]{\about{#1}}

% Ditto within an index entry:
\newcommand*{\predidx}[1]{\about{#1}\index{#1@\pred{#1}}}

% Ditto without putting into the text:
\newcommand*{\predidxonly}[1]{\index{#1@\pred{#1}}}

% An index entry for a procedure:
\newcommand*{\progidx}[1]{\prog{#1}\index{#1@\prog{#1}}}

% Ditto without putting into the text:
\newcommand*{\progidxonly}[1]{\index{#1@\prog{#1}}}

% A reference to a figure:
\newcommand*{\Figref}[1]{Fig.~\vref{#1}}

% A reference to a chapter:
\newcommand*{\Chapref}[1]{chapter~\ref{#1}}

% A reference to a section:
\newcommand*{\Secref}[1]{Sec.~\protect\ref{#1}}

% A reference to a page:
\newcommand*{\Pageref}[1]{p.~\pageref{#1}}

% An indentation
\newcommand*{\ind}{\hbox{\hspace{2em}}}


%----------------------------------------------------------------------
%  ENVIRONMENTS

% A warning:
\newenvironment{Warning}%
{\begin{quote}\textbf{Warning:}\itshape}%
{\end{quote}}

% Itemize with no topsep:
\newenvironment{Itemize}%
{\begin{list}{$\bullet$}%
    { \setlength{\topsep}{0pt}%
}}%
{\end{list}}

% A lightweight itemize:
\newenvironment{LightItemize}%
{\begin{list}{--}%
    { \setlength{\itemsep}{0.1ex}%
      \setlength{\topsep}{0pt}%
}}%
{\end{list}}

% Enumerate with no topsep:
\newcounter{Enumcnt}
\newenvironment{Enumerate}
               {\begin{list}{\arabic{Enumcnt}.}%
                   { \setlength{\topsep}{0pt}%
                     \usecounter{Enumcnt}%
               }}%
               {\end{list}}

% A lightweight enumerate, labeled with (a), {b):
\newcounter{lightenumcnt}
\newenvironment{LightEnumerate}
               {\begin{list}{(\alph{lightenumcnt})}%
                   {\setlength{\itemsep}{0pt}%
                     \setlength{\topsep}{0pt}%
                     \settowidth{\labelwidth}{(m)}%
                     \usecounter{lightenumcnt}%
               }}%
               {\end{list}}

%----------------------------------------------------------------------
\title{The DRA Interpreter\\
User Manual}

\author{Feliks Klu{\'z}niak\\
  \emph{Applied Logic, Programming Languages and Systems Lab}\\
  \emph{Department of Computer Science}\\
  \emph{University of Texas at Dallas}
}
\date{\small\today}

\bibliographystyle{plain}


%----------------------------------------------------------------------
\begin{document}

\maketitle

%%% reverse of title page
\thispagestyle{empty}
\setcounter{page}{0}

\copyright 2009 University of Texas at Dallas
\vfill

{\footnotesize
  All comments, queries and suggestions about this manual or the software
  are welcome. The author's e-mail address is
  \prog{feliks.kluzniak@utdallas.edu}.}
%%% end reverse of title page

\tableofcontents

\Chapter{Introduction\label{sec:intro}}

\Section{Overview\label{sec:overview}}

This document is a user manual for \about{dra}, an interpreter for tabled
logic programming with coinduction.  The manual has been written for users of
various versions of Unix: if you are running another system, various small
details (such as the way to enter the end-of-file character) may be
different.

The interpreter implements ``top-down tabled programming'' via so called
``Dynamic Reordering of Alternatives'' \cite{guo-gupta-dra}.  It also
supports ``co-logic programming'', i.e., logic programs that contain
coinductive predicates \cite{coinductive}, \cite{co-LP}.


Apart from support for coinduction, there are two significant changes with
respect to the original description \cite{guo-gupta-dra}:
\begin{Enumerate}

\item
  A tabled goal will never produce the same result twice.\footnote{
  In this document \Defconcept{goal} means an instance of a procedure call
  (i.e., an invocation of a predicate).  A \Defconcept{tabled
    goal}\index{goal!tabled} is a goal that invokes a tabled predicate. A
  \Defconcept{result} is the instantiation of a goal when it succeeds.}

  More precisely: a tabled goal will not succeed twice with instantiations
  that are variants of each other.\index{duplicate
    results}\index{result!duplicate}

\item
  By default, new results for a tabled goal will be produced before old
  answers.\index{order of results}\index{result!order}
  The user can reverse the order by issuing the directive\\
  \ind\prog{:-~old\_first.}\label{dir:old-first}\progidxonly{old\_first}

  A ``new result for a tabled goal''\label{new-result}\index{new
    result}\index{result!new} is a result that has not yet been tabled for
  this goal.  (More precisely: a result such that the table does not yet
  contain a variant of this result associated with a variant of this goal.)

  The default behaviour is intended to help computations converge more
  quickly.  The user is given an option to change it, because some predicates
  may produce a very large (even infinite) set of answers on backtracking,
  and the application might not require those answers.
\end{Enumerate}


%-------------------------------------------------------------------------------
\Section{Tabling\label{sec:tabling}}%
\index{tabled predicate}%
\index{predicate!tabled}

By default, each predicate that is defined in the interpreted program is a
``normal'' Prolog predicate.  If you want a predicate (\pred{p/2}, say) to be
tabled, you must put an appropriate declaring directive in your
program:%
\footnote{
    In literature about logic programming with tabling one often
    encounters the form \prog{:- table p/2}.  Since the directive is
    almost universally called a declaration (and not a command), the word
    \about{tabled} is obviously more appropriate.}\\
\ind\prog{:-~tabled~p/2.}\label{dir:tabled}\progidxonly{tabled}

When you declare a predicate as tabled, there will be
two consequences:
\begin{Enumerate}
\item
  Every time a goal with this predicate symbol (i.e., an invocation of this
  predicate) succeeds, the result (i.e., the resulting instantiation of the
  goal) is stored in a special table called the \Defconcept{answer
    table}\index{table|see{answer table}}.  Such a tabled result is commonly
  referred to as an \Defconcept{answer}.\index{answer}

  Once all the possible results have been so stored, subsequent invocations
  can draw upon the tabled answers, instead of recomputing them all over
  again. This is all done in a way that does not affect the semantics of the
  program, except perhaps for the order in which results are reported and the
  number of repetitions of the same result. (Of course, if your program is
  not sufficiently close to pure Prolog, e.g., if it relies on and/or
  produces side-effects, then all bets are off.)

\item
  When a goal that invokes a tabled predicate is found to be a variant of one
  of its ancestors on the recursion stack, the goal is not
  expanded.\footnote{
    Instead, it succeeds with results that have already been put into the
    table (or fails if there are none).  Things are arranged so that no
    results are lost.}
  This allows you to write your predicates in a more declarative fashion,
  without worrying, e.g., about the danger of left recursion.
\end{Enumerate}


Tabling is not always as advantageous as one might think, for two reasons:
  \begin{Enumerate}
  \item
    If the number of results is large, and goals are seldom repeated, the
    main effect might be a significantly increased demand for memory, which
    is not offset by a shorter processing time.

  \item It so happens that in order to preserve the semantics of the program
    the tabled answers must be associated with the goal that produced them,
    and are accessible only to variant goals.  Other goals that invoke the
    same predicate will require recomputation, and the computed results will
    increase the size of the table, even if they are the same as those that
    have already been stored.
  \end{Enumerate}


A more extensive discussion of tabling is well outside the scope of this user
manual.



%-------------------------------------------------------------------------------
\Section{Coinduction\label{sec:coindution}}%
\index{coinductive predicate}%
\index{predicate!coinductive}

The interpreter supports ``old style'' coinduction (see \cite{coinductive}
and \cite{co-LP}), as well as ``new style'' coinduction
\cite{gupta-coinductive-personal}.

If you want a predicate (\pred{p/2}, say) to be treated as an ``old style''
coinductive predicate, you must put the following declaring directive in your
program:\\
\ind\prog{:-~coinductive~p/2.}%
\label{dir:coinductive}\progidxonly{coinductive}

Such a declaration has one effect:
\begin{Itemize}
  \item
    If a goal that invokes a coinductive predicate is found to be unifiable
    with an ancestor on the recursion stack, the goal is unified with the
    ancestor and succeeds.  Upon backtracking it is unified---one by
    one\footnote{
      In \about{dra} this is done in reverse chronological order, i.e.,
      the more immediate ancestors are used first.
    }---with other such ancestors;
    it is expanded in the normal way (i.e., by
    using clauses) only when all the unifiable ancestors have been taken
    advantage of in this manner.
\end{Itemize}

If you want \pred{p/2} to be treated as a ``new style'' coinductive predicate,
use the declaration:\\
\ind\prog{:-~coinductive1~p/2.}%
\label{dir:coinductive1}\progidxonly{coinductive1}\\
This causes the predicate to be treated as above, except that clauses are used
only when there are no unifiable ancestors.\footnote{%
  The mnemonic value of our particular convention is that the ``new style''
  coinductive predicates give you only one way to satisfy a goal.}

The declaration \prog{:-~coinductive~p/2.} subsumes \prog{:-~coinductive1~p/2.}
if both are present in the same program.  If a predicate is declared as both
coinductive in the new style and tabled, then tabled answers are used even for
goals that have unifiable ancestors.

Declaring a predicate as coinductive gives it a radically new meaning, which is
often the appropriate one for operations on ``infinite'' terms (represented by
cyclic terms).  Again, further discussion of the concept is outside the scope of
this document.

\Chapter{The interpreted programs\label{chap:programs}}



%-------------------------------------------------------------------------------
\Section{Limitations\label{sec:limitations}}

The interpreter does not support full Prolog.  Here are the main limitations
of the interpreted language:
\begin{enumerate}

\item
  The interpreted program must not contain cuts.  Use of the conditional
  construct is permitted, as is the use of \pred{once/1}.

\item
  The interpreted program must not contain variable literals.  It may contain
  invocations of \pred{call/1}, but if the argument of \pred{call/1} is not
  properly instantiated at runtime, the results will be undefined.

  In some cases the interpreter can verify that the argument of an occurrence
  of \pred{call/1} cannot be instantiated at run-time, and it will then raise
  a fatal error.  The check is quite conservative, so the absence of such an
  error message does not mean that the program is correct in that respect.

\item
  The repertoire of built-in procedures recognized by the interpreter is
  somewhat limited.  This is done by design, to facilitate porting to
  different Prolog systems.  The recognized built-ins are declared in the
  file \prog{dra\_builtins.pl}, and new built-ins can be added as the need
  arises.  For most built-ins it is a matter of just adding another line to
  the file, but a few might require special treatment by the
  interpreter.\footnote{
    Please contact me if you have question or encounter any problems.}
\end{enumerate}

If those limitations seem too strict, you may in some cases get around them
by separating your program into two layers: see \Secref{sec:support}.


%-------------------------------------------------------------------------------
\Section{The notion of ``support''\label{sec:support}}

The interpreter provides you with an opportunity to divide your program into
two layers: an upper layer which makes use of the special facilities provided
by the interpreter (i.e., tabling and/or coinduction), and a lower layer of
``support'' software that requires only standard Prolog.  This can be useful
for increasing efficiency: the support layer will be compiled just as all
other ``normal'' Prolog programs.  An additional advantage is that the
support layer can use the full range of built-in predicates available in the
host logic programming system.

The interface between the two layers consists of a handful of entry-point
predicates, each of which is  declared by a directive similar to the
following one:\\
\ind\prog{:- support check\_consistency/1.}\progidx{support}\\
Please note that this directive cannot be entered interactively: it must be
included in the text of the upper layer part of your program.

The support declaration means that the metainterpreter should treat the
declared predicate as a built-in, i.e., just let Prolog execute it.

The support layer cannot invoke the upper layer, so there is no need to
declare those predicates in the support layer that are not directly invoked
by the upper layer.

Predicates that are declared as "support" must be defined in other files.  To
compile and load such files, use the following directive in the text of your
program:\\
\ind\prog{:- load\_support(~}\patt{filename}\prog{~).}\\

In this context, the extension of the \patt{filename} will usually be your
logic programming system's default extension for names of files that contain
Prolog code.%
\index{default extension (file name)}%
\index{extension of file name!default}\index{file!name!default extension}



%-------------------------------------------------------------------------------
\Section{Directives\label{sec:directives}}

Apart from the support directive and directives for file inclusion (see
\Secref{sec:including}, the interpreter recognizes a number of other
directives, which are listed below.  Any other directive is treated as an
error.
\begin{itemize}

\item [Declarations of tabled predicates]
  To declare a predicate as tabled, use a directive similar to the following:\\
  \ind\prog{:-~tabled~ancestor/2.}\progidx{tabled}\\
  The declaration should precede the definition of the predicate.\footnote{
    In literature about logic programming with tabling one can often
    encounter the form \prog{:- table~}\patt{...}.  Since the directive is
    almost universally called a declaration (i.e., not a command), the word
    \about{tabled} is obviously more appropriate.}

\item [Declarations of coinductive predicates]
  To declare a predicate as tabled, use a directive similar to the following:\\
  \ind\prog{:-~coinductive~ancestor/2.}\progidx{coinductive}\\
  The declaration should precede the definition of the predicate.

\item [Declarations of ``entry points'']
  Before execution begins, the program is subjected to a number of useful
  sanity checks.  One of these is a check whether every defined predicates
  is actually called from somewhere (i.e., whether there is no dead code).

  Since it is not unusual for a program to contain a handful of such
  predicates (intended as ``entry points'' that are supposed to be invoked
  from a query), the user can declare them by using a directive similar to
  the following:\\
  \ind\prog{:-~top~p/1,~q/2.}\progidx{top}\\
  The declaration is given only to suppress warnings.  However, it is an
  error for an undefined predicate or a support predicate to be so declared.

\item[Declarations of dynamic predicates]
  To declare a predicate whose clauses are asserted and/or retracted by the
  interpreted program, use\\
  \ind\prog{:-~ dynamic~p/k.}\progidx{dynamic}

\item[Changing the order of results\label{old-first}]
  By default, a goal produces new (i.e., heretofore unknown) results before
  producing old ones (see also \Pageref{new-result}).  To reverse this
  behaviour,  use\\
  \ind\prog{:-~old\_first~p/k}.\progidx{old\_first}\\
  or\\
  \ind{:-~old\_first~all.}

\item[Producing a trace]
  To produce a wallpaper trace of what happens to some chosen predicates, use
  a directive similar to the following:\\
  \ind\prog{:-~trace~p/3,~q/0,~r/1.}\progidx{trace}\\
  If you want to trace everything, use\\
  \ind\prog{:-~trace~all.}\\
  These directives are cumulative.

  All the details of the trace will not be easy to understand for someone who
  does not know the details of the DRA method \cite{guo-gupta-dra}, but you
  can glean some useful information, especially about what new answers are
  added to the table.

\item[Querying the answer table]
To print out subsets of the current answer table, use\\
\ind\prog{:-~answers(~}\patt{Goal}\prog{,~}\patt{Pattern}\prog{~).}%
\progidx{answers}\\
where \patt{Goal} and \patt{Pattern} are terms.
This will print all tabled answer that are associated with a variant of the
goal and unifiable with the pattern.

To produce a dump of the entire table, use just\\
\ind\prog{:-~answers.}

\item[Producing a trace]
  To produce a wallpaper trace of what happens to some chosen predicates, use
  a directive similar to the following:\\
  \ind\prog{:-~trace~p/3,~q/0,~r/1.}\progidx{trace}\\
  If you want to trace everything, use\\
  \ind\prog{:-~trace~all.}\\
  These directives are cumulative.

  All the details of the trace will not be easy to understand for someone who
  does not know the details of the DRA method \cite{guo-gupta-dra}, but you
  can glean some useful information, especially about how new answers are
  added to the table.

\end{itemize}


%-------------------------------------------------------------------------------
\Section{hooks\label{sec:hooks}}

The program may contain clauses that modify the definition of the
interpreter's predicate \pred{essence\_hook/2} (the clauses will be asserted
at the front of the predicate, and will thus override the default definition
for some cases).  The interpreter's default definition is\\
\ind\prog{essence\_hook(~T,~T~).}

This predicate is invoked, in certain contexts, when:
\begin{LightItemize}
  \item
    two terms are about to be compared (either for equality or to check
    whether they are variants of each other);
  \item
    an answer is tabled;
  \item
    an answer is retrieved from the table.
\end{LightItemize}

The primary intended use is to suppress arguments that carry only
administrative information and that may differ in two terms that are
``semantically'' equal or variants of each other.

For example, the presence of\\
\ind\prog{essence\_hook(~p(~A,~B,~\_~),~~p(~A,~B~)~).}\\

will result in \prog{p(~a,~b,~c~)} and \prog{p(~a,~b,~d~)} being treated as
identical, as each of them will be translated to \prog{p(~a,~b~)} before
comparison.

\begin{Warning}
This facility should be used with the utmost caution, as it may drastically
affect the semantics of the interpreted program in a fashion that would be
hard to understand for someone who does not understand the details of the
interpreter.
\end{Warning}

\Chapter{Running a program\label{chap:running}}


%-------------------------------------------------------------------------------
\Section{Loading the interpreter\label{sec:loading-dra}}%
\index{loading the interpreter}

The interpreter is written in Prolog.  It is distributed in source
form.%
%\footnote{
%  Please see the ``README'' files in the distribution tree: they will help you
%  find your way around.}

The interpreter is known to run on Eclipse 6.0 and Sicstus 4.0.  If you plan
to run programs that take advantage of coinductive programming, you might
prefer to use Sicstus, which has much better support for cyclic terms.

For both these systems, the simplest way to proceed is to:
\begin{Enumerate}
\item
  start your logic programming system;
\item
  type in the following directive:\\
  \ind\prog{:-~[~'\patt{Path}/tabling/dra'~].}\\
  where \about{Path} is the path to the root of the distribution
  tree.\footnote{
    This will just load the interpreter, but you will still be interacting
    with the host logic programming system.  \Secref{sec:loading-prog}
    describes how to start the interpreter.}
\end{Enumerate}

The interpreter is encapsulated in its own module, called \about{dra}.  So if
you are running Eclipse, you will probably find it more convenient to import
the module by writing\\
\ind\prog{:- import dra.}\label{import-dra}\progidxonly{import}\\
immediately after loading the interpreter.

It may well be that things have been installed differently on your site.
This might be because the interpreter has been modified to run with a
different Prolog system, or because an immediately-loadable version has been
made available in some standard directory. The person responsible for the
local installation of the interpreter will provide you with more details.


%-------------------------------------------------------------------------------
\Section{Loading a program\label{sec:loading-prog}}%
\index{loading a program}

Once you have loaded the interpreter into your logic programing system, you
may want to load and run a program in the interpreter. This is done by
writing\\
\ind\prog{prog(~\patt{filename}~).}\progidxonly{prog}%
\footnote{
  If you are running in Eclipse, and have not imported the module \about{dra}
  (as explained on \Pageref{import-dra}), you must write \prog{dra:prog}
  instead of \prog{prog}.
}\\
\patt{filename} should be the name of the file that contains your program.
If the name is given with no extension, it will be automatically extended
with \prog{.tlp}.%
\index{default extension}%
\index{extension of file name!default}\index{file!name!default extension}
If the name should have a different extension, you must type in the entire
name, enclosed in single quotes, e.g.,\\
\ind\prog{prog(~'myfile.pl'~).}\\
Quotes must also be used if the file is not in the current directory and you
are providing an absolute or relative path.

As the file is being read and loaded, directives and queries are interpreted
on-the-fly. Each query is evaluated to give all solutions (i.e., as if the
user kept responding with a semicolon): to avoid that you can use the
built-in predicate \predidx{once/1} in the queries.

You should be aware that loading a program obliterates all traces of
previously loaded programs, including the contents of the answer table.  If
you are interested in re-running your program from scratch (so that it does
not take advantage of answers that were already tabled), you can just load it
again.

\vfill %<<<<<<<<<<<<<<<<<<<<<<<<<<<<<<


%-------------------------------------------------------------------------------
\Section{Interacting with a loaded program\label{sec:interacting}}

%%%
\Subsection{The interactive mode\label{sec:interactive-mode}}%
\index{interactive mode}

After the file is loaded (and all the directives and queries it contains are
executed), the interpreter enters interactive mode.  This is very much like
the usual top-level loop, except that it is the interpreter---and not the
underlying logic programming system---that evaluates queries and executes
directives.

In the interactive mode the interpreter will read your input and act on it.
Input consists of a term, terminated by a fullstop and followed by a newline
(i.e., you must press the \prog{ENTER} key). You cannot input more than one
term per line: all text between the fullstop and the newline will be ignored.

When you type in a term of the form \prog{:-~\patt{...}.}, it will be treated
as a directive\index{directive}; when you type in a term of the form
\prog{?-~\patt{...}.}, it will be treated as a query\index{query}; when you
type in a term that does not begin with \prog{:-} or \prog{?-}, it will also
be treated as a query.

The difference between directives and queries is quite crucial, because the
names of the directives do not occupy the same name space\index{name space}
as the names of predicates.  If you type in, say,\\
\ind\prog{answers(~\_,~\_~).}\\
this will have nothing to do with the directive\\
\ind\prog{:-~answers(~\_,~\_~).}\\
and the interpreter will try to invoke the predicate \pred{answers/2} in your
program.  This may be a little confusing, but the good news is that you don't
have to worry about potential conflicts between the names in your program and
the names of the interpreter's directives.  (Neither do you have to worry
about conflicts between your program and the interpreter itself.)\footnote{
  The interpreted program is loaded into a separate module
  called \progidx{interpreted}.  If there is a support layer, it is loaded
  into the module \progidx{support}.  I mention these names, because the host
  system may show them in error messages if something goes horribly wrong.
}

\vfill %<<<<<<<<<<<<<<<<<<<<<<<<<<<<<<


%%%%
\Subsection{Resuming the interactive mode\label{sec:resuming-interactive}}%
\index{interactive mode!resuming}

To just enter interactive mode (without loading a new program)
invoke\footnote{
  Again, \prog{dra:top} in Eclipse, if you have not imported\about{dra}.}\\
\ind\prog{top.}\progidxonly{top}

The interpreter does not allow you to input clauses directly from your
terminal, but it's good to have recourse to this call if you have exited
interactive mode (see below) or if the execution of the interpreter was
interrupted (either because of a fatal error, or because you pressed Ctrl-C
on your keyboard). The program that was most recently loaded is still there,
the answer table might have been populated, so you might want to resume
interactive mode.


%%%%
\Subsection{Exiting the interactive mode\label{sec:exiting-interactive}}%
\index{interactive mode!exiting}

To exit the interactive mode enter the end of file character
(\about{Ctrl-D}),%
\footnote{
  \about{Ctrl-D} appears not to work with tkeclipse.}
or just write\\
\ind\prog{quit.}\progidxonly{quit}


%%%%
\Subsection{Statistics\label{sec:statistics}}%
\index{statistics}

Just before the result of a query is reported, the interpreter produces a
printout with statistics\index{statistics} accumulated since the previous
such printout (or since the beginning, if this is the first printout during
the current session with the interpreted program). The printout looks like
this:\\
\ind\prog{[\patt{K}~steps,~\patt{M}~new~answers~tabled~(\patt{N}~in~all)]}\\
\patt{K},\patt{M} and \patt{N} are natural numbers. \patt{K} is the number of
evaluated goals, \patt{M} is the number of new additions to the answer table,
and \patt{N} is the current size of the answer table.

Please note that you might sometimes see new answers tabled in 0 steps: this
may happen when you ask for more results (by typing a semicolon) and the last
goal to be activated has still not completed its task.


%%%%
\Subsection{Print depth\label{sec:print-depth}}%
\index{print depth}

When a query succeeds, the instantiations of its variables should be printed
upto a certain maximum depth.  The default value in the distributed version
of the interpreter is 10.  The maximum depth can be changed from the
interpreted program (or interactively from the top-level) by invoking\\
\ind\prog{set\_print\_depth(~\patt{N}~)}\predidxonly{set\_print\_depth/1}\\
where \patt{N} is a positive integer.

Please note that with some Prolog implementations this might not prevent a
loop if the printed term is cyclic (as will often happen for coinductive
programs).

Note also that the foregoing does not apply to invocations of built-in
predicates in the interpreted program.  It is up to the user to apply the
built-in that is appropriate for the host logic programming system.  For
example, in the case of Sicstus, use
\prog{write\_term(~T,~[~max\_depth(~10~)~]~)}, rather than just \prog{write(
  T )}, if you expect the instantiation of \prog{T} to be cyclic.



%-------------------------------------------------------------------------------
\Section{Including other files\label{sec:including}%
\index{including a file}\index{file!inclusion}}

To include files (interactively or from other files) you can use the usual
Prolog syntax:\\
\ind
\prog{:-~[~\patt{filename1},~\patt{filename2},~\patt{...}~].}%
\label{dir:include}\\
The default extension is \prog{.tlp}.%
\index{default extension}%
\index{extension of file name!default}\index{file!name!default extension}

Please note that including a file with \prog{:-~[~\patt{filename}~].}  and
loading a program with \prog{prog(~\patt{filename}~).} are very different
actions. When the interpreter includes a file, the contents are just
added to its memory. When it loads a program, it first (re)initializes
itself, wiping out the previously loaded program, all included files and the
answer table.



%-------------------------------------------------------------------------------
\Section{Inspecting the answer table\label{sec:answer-table}}%
\index{answer table}

In principle, the answer table is an auxiliary data structure that is, in
effect, accessed by normal queries.

However, the interpreter gives you the possibility of looking ``under the
hood'' by accessing the table directly.  This might be useful for assessing
the efficacy of your tabling declarations, or simply for satisfying your
curiosity.

To print out subsets of the current answer table, use\\
\ind\prog{:-~answers(~\patt{Goal},~\patt{Pattern}~).}%
\label{dir:answers}\progidxonly{answers}\\
where \patt{Goal} and \patt{Pattern} are terms.
This will print all those tabled answers that are associated with a variant
of the goal and unifiable with the pattern.  If the first argument is a
variable, the pattern will be used as a filter for all the answers in the
table.

To produce a dump of the entire table, just use\\
\ind\prog{:-~answers(~\_,~\_~).}



%-------------------------------------------------------------------------------
\Section{The ``wallpaper'' trace\label{sec:walpaper-trace}}

The interpreter does not incorporate an interactive debugger, but it can
produce a long trace of what happens during the execution of an interpreted
program.  This facility is useful mainly for helping to diagnose problems
with the interpreter: some of the information in the trace will not be easy
to understand for someone who does not know the details of the DRA
method~\cite{guo-gupta-dra}, and I will not try to explain it all here.
Still, you might sometimes be able to get some useful information from the
trace, e.g, about how new answers are added to the table.

To produce a wallpaper trace of what happens to some chosen predicates, use a
directive similar to the following:\\
\ind\prog{:-~trace~p/3,~q/0,~r/1.}\label{dir:trace}\progidxonly{trace}\\
If you want to trace all predicates, use\\
\ind\prog{:-~trace~all.}\\
These directives are cumulative.



\chapter*{Summary of directives\label{directives}}%
\addcontentsline{toc}{chapter}{Summary of directives}


An argument specified as \patt{PredSpec} can take three forms:
\begin{LightEnumerate}
\item
  A predicate specification written as \pred{name/arity}: for example
  \prog{foo/3} (in the short descriptions below we will assume this is the form
  that is used);
\item
  A sequence of such specifications, separated by commas: for example
  \prog{p/2,~q/1,~r/3};
\item
  The word \prog{all}, which specifies all predicates. (This cannot be used
  for \prog{support} and \prog{dynamic}!)
\end{LightEnumerate}
If the same kind of directive occurs a number of times, specifying different
predicates, the results are cumulative.  In particular, \prog{all} subsumes all
other predicate specifications.

\newlength{\DescWidth}
\setlength{\DescWidth}{16em}
\begin{tabular}{llr}
\emph{Directive:}      & \emph{Short description:}   \\

\prog{:-~[~\patt{filename}~].}
                   & \parbox[t]{\DescWidth}{
                        load a part of the program (\Pageref{dir:include})}\\

\prog{:-~answers(~\patt{Goal},~\patt{Pattern}~).}
                  & \parbox[t]{\DescWidth}{
                       inspect the answer table (\Pageref{dir:answers})}\\

\prog{:-~coinductive~\patt{PredSpec}.}
                   & \parbox[t]{\DescWidth}{
                       predicate is coinductive (\Pageref{dir:coinductive})}\\

\prog{:-~dynamic~\patt{PredSpec}.}
                  & \parbox[t]{\DescWidth}{
                        predicate is dynamic (\Pageref{dir:dynamic})}\\

\prog{:- load\_support(~\patt{filename}~).}
                   & \parbox[t]{\DescWidth}{
                      load (a part of) the support layer
                                                (\Pageref{dir:load-support})}\\

\prog{:-~old\_first.} &  \parbox[t]{\DescWidth}{
                                 change the order in which results are produced
                                 (\Pageref{dir:old-first})
                                 }\\

\prog{:- support~\patt{PredSpec}}
                   & \parbox[t]{\DescWidth}{
                       predicate is an entry point to the support layer
                                                   (\Pageref{dir:support})}\\

\prog{:-~tabled~\patt{PredSpec}.}
                   & \parbox[t]{\DescWidth}{
                            predicate is tabled (\Pageref{dir:tabled})}\\

\prog{:-~top~\patt{PredSpec}}
                   & \parbox[t]{\DescWidth}{
                        predicate is an entry point (\Pageref{dir:top})}\\

\prog{:-~trace~\patt{PredSpec}.}
                  & \parbox[t]{\DescWidth}{
                        trace the predicate (\Pageref{dir:trace})}
\end{tabular}


\newpage
\bibliography{bibliography}

\newpage
\printindex
\end{document}
