\Chapter{Starting the interpreter\label{chap:starting}}

The interpreter is written in Prolog.  It is distributed in source
form.%
\footnote{
  Please see the ``README'' files in the distribution tree: they will help you
  find your way around.}

The interpreter is known to run on Eclipse 6.0 and Sicstus 4.0.  If you plan
to run programs that take advantage of coinductive programming, you might
prefer to use Sicstus, which has much better support for cyclic terms.

For both these systems, the simplest way to proceed is:
\begin{enumerate}
\item
  start your logic programming system;
\item
  type in the following directive:\\
  \ind\prog{:-~[~'}\about{Path}\prog{/tabling/dra'~]}\\
  where \about{Path} is the path to the root of the distribution tree.
\end{enumerate}

The interpreter is encapsulated in its own module, called \about{dra}.  So if
you are running Eclipse, you will probably find it more convenient to import
the module by writing\\
\ind\prog{:- import dra.}\label{import-dra}\progidx{import}

It may well be that things have been installed differently on your site.
This might be because the interpreter has been modified to run with a
different Prolog system, or because an immediately-loadable version has been
made available in some standard directory. The person responsible for the
local installation of the interpreter will provide you with more details.
