\Chapter{The interpreted programs\label{chap:programs}}



%-------------------------------------------------------------------------------
\Section{Limitations\label{sec:limitations}}

The interpreter does not support full Prolog.  Here are the main limitations
of the interpreted language:
\begin{Enumerate}

\item
  The interpreted program must not contain cuts\index{cut}.  Use of the
  conditional construct\index{conditional construct} is permitted, as is the
  use of \predidx{once/1}.

\item
  The interpreted program must not contain variable literals.  It may contain
  invocations of \predidx{call/1}, but if the argument of \pred{call/1} is
  not properly instantiated at runtime, the results will be undefined.

  In some cases the interpreter can verify that the argument of an occurrence
  of \pred{call/1} cannot be instantiated at run-time, and it will then raise
  a fatal error.  The check is quite conservative, so the absence of such an
  error message does not mean that the program is correct in that respect.

\item
  The repertoire of built-in predicates\index{built-in predicates} recognized
  by the interpreter is somewhat limited.  This is done by design, to
  facilitate porting to different Prolog systems.  The recognized built-ins
  are declared in the file \prog{dra\_builtins.pl}, and new declarations can be
  added as the need arises.  For most built-ins it is a matter of just adding
  another line to the file, but a few might require special treatment by the
  interpreter.
\end{Enumerate}

If those limitations seem too strict, you may in some cases get around them
by separating your program into two layers: see \Secref{sec:support}.


%-------------------------------------------------------------------------------
\Section{The notion of ``support''\label{sec:support}}

The interpreter provides you with an opportunity to divide your program into
two layers: an upper layer which makes use of the special facilities provided
by the interpreter (i.e., tabling and/or coinduction), and a lower layer of
``support'' software that requires only standard Prolog.  This can be useful
for increasing efficiency: the support layer will be compiled just as all
other ``normal'' Prolog programs.  An additional advantage is that the
support layer can use the full range of built-in predicates available in the
host logic programming system.

The interface between the two layers consists of a handful of entry-point
predicates, each of which is  declared by a directive similar to the
following one:\\
\ind\prog{:- support check\_consistency/1.}%
\label{dir:support}\progidx{support}\\
Please note that this directive cannot be entered interactively: it must be
included in the text of the upper layer part of your program.

The support declaration means that the metainterpreter should treat the
declared predicate as a built-in, i.e., just let Prolog execute it.

The support layer cannot invoke the upper layer, so there is no need to
declare those predicates in the support layer that are not directly invoked
by the upper layer.

Predicates that are declared as "support" must be defined in other files.  To
compile and load such files, use the following directive in the text of your
program:\\
\ind\prog{:- load\_support(~\patt{filename}~).}%
\label{dir:load-support}\progidx{load\_support}\\
In this context, the default extension of the \patt{filename} will be the
default extension used by the host logic programming system for names
of files that contain Prolog code.%
\index{default extension (file name)}%
\index{extension of file name!default}% \index{file!name!default extension}



%-------------------------------------------------------------------------------
\Section{Declaration of entry points\label{sec:entry-points}}

Before execution begins, the interpreted program is subjected to a number of
useful sanity checks.  One of these is a check whether every defined
predicate is actually called from somewhere (i.e., whether there is no dead
code).

Since it is not unusual for a program to contain a handful of such predicates
(intended as ``entry points'' that are to be invoked from a query),
the user can declare them by using a directive similar to the following:\\
\ind\prog{:-~top~p/1,~q/2.}\label{dir:top}\progidx{top}\\
The declaration is given only to suppress warnings.  However, it is an error
for an undefined predicate or a support predicate to be so declared.



%-------------------------------------------------------------------------------
\Section{Inspecting the answer table\label{sec:answer-table}}

In principle, the answer table is an auxiliary data structure that is, in
effect, accessed by normal queries.

However, the interpreter gives you the possibility of looking ``under its
hood'' by accessing the table directly.  This might be useful for assessing
the efficacy of your tabling declarations, or simply for satisfying your
curiosity.

To print out subsets of the current answer table, use\\
\ind\prog{:-~answers(~\patt{Goal},~\patt{Pattern}~).}%
\label{dir:answers}\progidx{answers}\\
where \patt{Goal} and \patt{Pattern} are terms.
This will print all tabled answer that are associated with a variant of the
goal and unifiable with the pattern.  If the first argument is a variable,
the pattern will be used as a filter for all the answers in the table.

To produce a dump of the entire table, just use\\
\ind\prog{:-~answers(~\_,~\_~).}

Please note that this sort of query must have the form of a directive.  This
is so even when you are interacting with the top level:  if you give it
just\\
\ind\prog{answers(~\_,~\_~).}\\
this will be treated as a query, equivalent to\\
\ind\prog{?-~answers(~\_,~\_~).}\\
and the interpreter will try to invoke the predicate \pred{answers/2} in your
program.  The good news is that nothing prevents you from having such a
predicate, i.e., the identifier \about{answers} is not reserved in any way.



%-------------------------------------------------------------------------------
\Section{The ``wallpaper'' trace\label{sec:walpaper-trace}}

The interpreter does not incorporate an interactive debugger, but it can
produce a long trace of what happens during the execution of an interpreted
program.  This facility is useful mainly for helping to diagnose problems
with the interpreter: some of the information in the trace will not be easy
to understand for someone who does not know the details of the DRA
method~\cite{guo-gupta-dra}, and I will not try to explain it all here.
Still, you might sometimes be able to get some useful information from the
trace, e.g, about how new answers are added to the table.

To produce a wallpaper trace of what happens to some chosen predicates, use a
directive similar to the following:\\
\ind\prog{:-~trace~p/3,~q/0,~r/1.}\label{dir:trace}\progidx{trace}\\
If you want to trace everything, use\\
\ind\prog{:-~trace~all.}\\
These directives are cumulative.


%-------------------------------------------------------------------------------
\Section{Declaring a dynamic predicate\label{sec:dynamic}}

To declare a predicate whose clauses are asserted and/or retracted by the
interpreted program, use\index{predicate!dynamic}\\
\ind\prog{:-~ dynamic~p/k.}\label{dir:dynamic}\progidx{dynamic}

\vfill %<<<<<<<<<<<<<<<<<<<<<<<<<<<<<<<<<<


%-------------------------------------------------------------------------------
\Section{Hooks\label{sec:hooks}}

The program may contain clauses that modify the definition of the
interpreter's predicate \pred{essence\_hook/2} (the clauses will be asserted
at the front of the predicate, and will thus override the default definition
for some cases).  The interpreter's default definition is\\
\ind\prog{essence\_hook(~T,~T~).}

This predicate is invoked, in certain contexts, when:
\begin{LightItemize}
  \item
    two terms are about to be compared (either for equality or to check
    whether they are variants of each other);
  \item
    an answer is tabled;
  \item
    an answer is retrieved from the table.
\end{LightItemize}

The primary intended use is to allow suppression of arguments that carry only
administrative information and that may differ in two terms that are
``semantically'' equal or variants of each other.

For example, the presence of\\
\ind\prog{essence\_hook(~p(~A,~B,~\_~),~~p(~A,~B~)~).}\\
will result in \prog{p(~a,~b,~c~)} and \prog{p(~a,~b,~d~)} being treated as
identical, as each of them will be translated to \prog{p(~a,~b~)} before
comparison.

\begin{Warning}
This facility should be used with the utmost caution, as it may drastically
affect the semantics of the interpreted program in a fashion that could be
hard to understand for someone who is not familiar with the details of the
interpreter.
\end{Warning}
