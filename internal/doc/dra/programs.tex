\Chapter{The interpreted programs\label{chap:programs}}



%-------------------------------------------------------------------------------
\Section{Limitations\label{sec:limitations}}

The interpreter does not support full Prolog.  Here are the main limitations
of the interpreted language:
\begin{Enumerate}

\item
  The interpreted program must not contain cuts\index{cut} (i.e., occurrences
  of \pred{!/2}\index{"!/2@\pred{"!/2}}).  Use of the conditional
  construct\index{conditional construct} is permitted, as is the use of
  \predidx{once/1}.

\item
  The interpreted program must not contain variable literals\index{variable
    literal}.  It may contain invocations of \predidx{call/1}, but if the
  argument of \pred{call/1} is not properly instantiated at runtime, you will
  get an error message and the interpreter will quit.\footnote{
    In some cases the interpreter can verify beforehand (i.e., at
    ``compile-time'') that the argument of an occurrence of \pred{call/1}
    cannot be instantiated at run-time, and it will then raise a fatal error.
    The check is quite conservative, so the absence of such an error message
    does not mean that the program is safe in that respect.}

\item
  The repertoire of built-in predicates\index{built-in predicates} recognized
  by the interpreter is somewhat limited.  This is done by design, mostly to
  facilitate porting to different Prolog systems.

  The recognized built-ins are declared in the file \prog{dra\_builtins.pl},
  and new declarations can be added as the need arises.  For most built-ins
  just adding another line to the file will suffice, but a few might
  require special treatment by the interpreter.\footnote{
    Having a file wherein you specify the names of built-in predicates you
    actually want to use does have its advantages.  Some logic programming
    systems (e.g., Eclipse) support a very extensive set of libraries that
    define built-in predicates whose names are treated as reserved even if
    you don't use the libraries.  As a result, many names that you might
    reasonably want to use in your programs are not available to you.}
\end{Enumerate}

If these limitations seem too strict, you may in some cases get around them
by separating your program into two layers: see \Secref{sec:support}.


%-------------------------------------------------------------------------------
\Section{The notion of ``support''\label{sec:support}}

The interpreter provides you with an opportunity to divide your program into
two layers: an upper layer which makes use of the special facilities provided
by the interpreter (i.e., tabling and/or coinduction), and a lower layer of
``support'' software that requires only standard Prolog.  This can be useful
for increasing efficiency: the support layer will be compiled just as all
other ``normal'' Prolog programs.  An additional advantage is that the
support layer can use the full range of built-in predicates available in the
host logic programming system, and in particular the cut.

The interface between the two layers consists of a handful of entry-point
predicates, each of which is  declared by a directive similar to the
following one:\\
\ind\prog{:- support check\_consistency/1.}%
\label{dir:support}\progidxonly{support}\\
Please note that this directive cannot be entered interactively: it must be
included in the text of the upper layer part of your program.

The support declaration means that the metainterpreter should treat the
declared predicate as a built-in, i.e., just let Prolog execute it.

The support layer cannot invoke the upper layer, so there is no need to
declare those predicates in the support layer that are not directly invoked
by the upper layer.

Predicates that are declared as ``support'' (and those that are---directly or
indirectly---called by them) must be defined in other files.
To compile and load such a file, use the following directive in the text of
your program:\\
\ind\prog{:- load\_support(~\patt{filename}~).}%
\label{dir:load-support}\progidxonly{load\_support}\\
In this context, the default extension of the \patt{filename} will be the
default extension used by the host logic programming system for names
of files that contain Prolog code.%
\index{default extension}%
\index{extension of file name!default}%
\index{file!name!default extension}



%-------------------------------------------------------------------------------
\Section{Declaring ``entry points''\label{sec:entry-points}}

Before execution begins, the interpreted program is subjected to a number of
sanity checks.  One of these is a check whether every defined predicate is
actually called from somewhere (i.e., whether there is no dead code).

Since it is not unusual for a program to contain a handful of such predicates
on purpose (they are intended as ``entry points'' that are to be invoked from
a query),
the user can declare them by using a directive similar to the following:\\
\ind\prog{:-~top~p/1,~q/2.}\label{dir:top}\progidxonly{top}\\
The declaration is given only to suppress warnings.  However, it is an error
for an undefined predicate or a support predicate to be so declared.



%-------------------------------------------------------------------------------
\Section{Declaring dynamic predicates\label{sec:dynamic}}

To declare a predicate whose clauses are asserted and/or retracted by the
interpreted program, use\index{predicate!dynamic}\\
\ind\prog{:-~ dynamic~p/k.}\label{dir:dynamic}\progidxonly{dynamic}



%-------------------------------------------------------------------------------
\Section{Hooks\label{sec:hooks}}\index{hook}

The program may contain clauses that modify the definition of the
interpreter's predicate \predidx{essence\_hook/2} (the clauses will be
asserted at the front of the predicate, and will thus override the default
definition for some cases).  The interpreter's default definition is\\
\ind\prog{essence\_hook(~T,~T~).}

This predicate is invoked, in certain contexts, when:
\begin{LightItemize}
  \item
    two terms are about to be compared (either for equality or to check
    whether they are variants of each other);
  \item
    an answer is tabled;
  \item
    an answer is retrieved from the table.
\end{LightItemize}

The primary intended use is to allow suppression of arguments that carry only
administrative information and that may differ in two terms that are
considered to be ``semantically'' equal or variants of each other.

For example, the presence of\\
\ind\prog{essence\_hook(~p(~A,~B,~\_~),~~p(~A,~B~)~).}\\
will result in \prog{p(~a,~b,~c~)} and \prog{p(~a,~b,~d~)} being treated as
identical: each of them will be translated to \prog{p(~a,~b~)} before
comparison.

\begin{Warning}
This facility should be used with the utmost caution, as it may drastically
affect the semantics of the interpreted program in a fashion that could be
hard to understand for someone who is not familiar with the details of the
interpreter.
\end{Warning}
