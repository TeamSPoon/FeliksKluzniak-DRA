\Chapter{Running a program\label{chap:running}}


%-------------------------------------------------------------------------------
\Section{Loading the interpreter\label{sec:loading-dra}}%
\index{loading the interpreter}

The interpreter is written in Prolog.  It is distributed in source
form.%
%\footnote{
%  Please see the ``README'' files in the distribution tree: they will help you
%  find your way around.}

The interpreter is known to run on Eclipse 6.0 and Sicstus 4.0.  If you plan
to run programs that take advantage of coinductive programming, you might
prefer to use Sicstus, which has much better support for cyclic terms.

For both these systems, the simplest way to proceed is to:
\begin{Enumerate}
\item
  start your logic programming system;
\item
  type in the following directive:\\
  \ind\prog{:-~[~'\patt{Path}/tabling/dra'~].}\\
  where \about{Path} is the path to the root of the distribution
  tree.\footnote{
    This will just load the interpreter, but you will still be interacting
    with the host logic programming system.  \Secref{sec:loading-prog}
    describes how to start the interpreter.}
\end{Enumerate}

The interpreter is encapsulated in its own module, called \about{dra}.  So if
you are running Eclipse, you will probably find it more convenient to import
the module by writing\\
\ind\prog{:- import dra.}\label{import-dra}\progidxonly{import}\\
immediately after loading the interpreter.

It may well be that things have been installed differently on your site.
This might be because the interpreter has been modified to run with a
different Prolog system, or because an immediately-loadable version has been
made available in some standard directory. The person responsible for the
local installation of the interpreter will provide you with more details.


%-------------------------------------------------------------------------------
\Section{Loading a program\label{sec:loading-prog}}%
\index{loading a program}

Once you have loaded the interpreter into your logic programing system, you
may want to load and run a program in the interpreter. This is done by
writing\\
\ind\prog{prog(~\patt{filename}~).}\progidxonly{prog}%
\footnote{
  If you are running in Eclipse, and have not imported the module \about{dra}
  (as explained on \Pageref{import-dra}), you must write \prog{dra:prog}
  instead of \prog{prog}.
}\\
\patt{filename} should be the name of the file that contains your program.
If the name is given with no extension, it will be automatically extended
with \prog{.tlp}.%
\index{default extension}%
\index{extension of file name!default}\index{file!name!default extension}
If the name should have a different extension, you must type in the entire
name, enclosed in single quotes, e.g.,\\
\ind\prog{prog(~'myfile.pl'~).}\\
Quotes must also be used if the file is not in the current directory and you
are providing an absolute or relative path.

As the file is being read and loaded, directives and queries are interpreted
on-the-fly. Each query is evaluated to give all solutions (i.e., as if the
user kept responding with a semicolon): to avoid that you can use the
built-in predicate \predidx{once/1} in the queries.

You should be aware that loading a program obliterates all traces of
previously loaded programs, including the contents of the answer table.  If
you are interested in re-running your program from scratch (so that it does
not take advantage of answers that were already tabled), you can just load it
again.

\vfill %<<<<<<<<<<<<<<<<<<<<<<<<<<<<<<


%-------------------------------------------------------------------------------
\Section{Interacting with a loaded program\label{sec:interacting}}

%%%
\Subsection{The interactive mode\label{sec:interactive-mode}}%
\index{interactive mode}

After the file is loaded (and all the directives and queries it contains are
executed), the interpreter enters interactive mode.  This is very much like
the usual top-level loop, except that it is the interpreter---and not the
underlying logic programming system---that evaluates queries and executes
directives.

In the interactive mode the interpreter will read your input and act on it.
Input consists of a term, terminated by a fullstop and followed by a newline
(i.e., you must press the \prog{ENTER} key). You cannot input more than one
term per line: all text between the fullstop and the newline will be ignored.

When you type in a term of the form \prog{:-~\patt{...}.}, it will be treated
as a directive\index{directive}; when you type in a term of the form
\prog{?-~\patt{...}.}, it will be treated as a query\index{query}; when you
type in a term that does not begin with \prog{:-} or \prog{?-}, it will also
be treated as a query.

The difference between directives and queries is quite crucial, because the
names of the directives do not occupy the same name space\index{name space}
as the names of predicates.  If you type in, say,\\
\ind\prog{answers(~\_,~\_~).}\\
this will have nothing to do with the directive\\
\ind\prog{:-~answers(~\_,~\_~).}\\
and the interpreter will try to invoke the predicate \pred{answers/2} in your
program.  This may be a little confusing, but the good news is that you don't
have to worry about potential conflicts between the names in your program and
the names of the interpreter's directives.  (Neither do you have to worry
about conflicts between your program and the interpreter itself.)\footnote{
  The interpreted program is loaded into a separate module
  called \progidx{interpreted}.  If there is a support layer, it is loaded
  into the module \progidx{support}.  I mention these names, because the host
  system may show them in error messages if something goes horribly wrong.
}

\vfill %<<<<<<<<<<<<<<<<<<<<<<<<<<<<<<


%%%%
\Subsection{Resuming the interactive mode\label{sec:resuming-interactive}}%
\index{interactive mode!resuming}

To just enter interactive mode (without loading a new program)
invoke\footnote{
  Again, \prog{dra:top} in Eclipse, if you have not imported\about{dra}.}\\
\ind\prog{top.}\progidxonly{top}

The interpreter does not allow you to input clauses directly from your
terminal, but it's good to have recourse to this call if you have exited
interactive mode (see below) or if the execution of the interpreter was
interrupted (either because of a fatal error, or because you pressed Ctrl-C
on your keyboard). The program that was most recently loaded is still there,
the answer table might have been populated, so you might want to resume
interactive mode.


%%%%
\Subsection{Exiting the interactive mode\label{sec:exiting-interactive}}%
\index{interactive mode!exiting}

To exit the interactive mode enter the end of file character
(\about{Ctrl-D}),%
\footnote{
  \about{Ctrl-D} appears not to work with tkeclipse.}
or just write\\
\ind\prog{quit.}\progidxonly{quit}


%%%%
\Subsection{Statistics\label{sec:statistics}}%
\index{statistics}

Just before the result of a query is reported, the interpreter produces a
printout with statistics\index{statistics} accumulated since the previous
such printout (or since the beginning, if this is the first printout during
the current session with the interpreted program). The printout looks like
this:\\
\ind\prog{[\patt{K}~steps,~\patt{M}~new~answers~tabled~(\patt{N}~in~all)]}\\
\patt{K},\patt{M} and \patt{N} are natural numbers. \patt{K} is the number of
evaluated goals, \patt{M} is the number of new additions to the answer table,
and \patt{N} is the current size of the answer table.

Please note that you might sometimes see new answers tabled in 0 steps: this
may happen when you ask for more results (by typing a semicolon) and the last
goal to be activated has still not completed its task.  You might also see
that new answers were added even though the final response is \prog{No}: this
only means that some auxiliary goals were successful, while the main one was
not.


%%%%
\Subsection{Print depth\label{sec:print-depth}}%
\index{print depth}

When a query succeeds, the instantiations of its variables should be printed
upto a certain maximum depth.  The default value in the distributed version
of the interpreter is 10.  The maximum depth can be changed from the
interpreted program (or interactively from the top-level) by invoking\\
\ind\prog{set\_print\_depth(~\patt{N}~)}\predidxonly{set\_print\_depth/1}\\
where \patt{N} is a positive integer.

Please note that with some Prolog implementations this might not prevent a
loop if the printed term is cyclic (as will often happen for coinductive
programs).

Note also that the foregoing does not apply to invocations of built-in
predicates in the interpreted program.  It is up to the user to apply the
built-in that is appropriate for the host logic programming system.  For
example, in the case of Sicstus, use
\prog{write\_term(~T,~[~max\_depth(~10~)~]~)}, rather than just \prog{write(
  T )}, if you expect the instantiation of \prog{T} to be cyclic.



%-------------------------------------------------------------------------------
\Section{Including other files\label{sec:including}%
\index{including a file}\index{file!inclusion}}

To include files (interactively or from other files) you can use the usual
Prolog syntax:\\
\ind
\prog{:-~[~\patt{filename1},~\patt{filename2},~\patt{...}~].}%
\label{dir:include}\\
The default extension is \prog{.tlp}.%
\index{default extension}%
\index{extension of file name!default}\index{file!name!default extension}

Please note that including a file with \prog{:-~[~\patt{filename}~].}  and
loading a program with \prog{prog(~\patt{filename}~).} are very different
actions. When the interpreter includes a file, the contents are just
added to its memory. When it loads a program, it first (re)initializes
itself, wiping out the previously loaded program, all included files and the
answer table.



%-------------------------------------------------------------------------------
\Section{Inspecting the answer table\label{sec:answer-table}}%
\index{answer table}

In principle, the answer table is an auxiliary data structure that is, in
effect, accessed by normal queries.

However, the interpreter gives you the possibility of looking ``under the
hood'' by accessing the table directly.  This might be useful for assessing
the efficacy of your tabling declarations, or simply for satisfying your
curiosity.

To print out subsets of the current answer table, use\\
\ind\prog{:-~answers(~\patt{Goal},~\patt{Pattern}~).}%
\label{dir:answers}\progidxonly{answers}\\
where \patt{Goal} and \patt{Pattern} are terms.
This will print all those tabled answers that are associated with a variant
of the goal and unifiable with the pattern.  If the first argument is a
variable, the pattern will be used as a filter for all the answers in the
table.

To produce a dump of the entire table, just use\\
\ind\prog{:-~answers(~\_,~\_~).}



%-------------------------------------------------------------------------------
\Section{The ``wallpaper'' trace\label{sec:walpaper-trace}}

The interpreter does not incorporate an interactive debugger, but it can
produce a long trace of what happens during the execution of an interpreted
program.  This facility is useful mainly for helping to diagnose problems
with the interpreter: some of the information in the trace will not be easy
to understand for someone who does not know the details of the DRA
method~\cite{guo-gupta-dra}, and I will not try to explain it all here.
Still, you might sometimes be able to get some useful information from the
trace, e.g, about how new answers are added to the table.

To produce a wallpaper trace of what happens to some chosen predicates, use a
directive similar to the following:\\
\ind\prog{:-~trace~p/3,~q/0,~r/1.}\label{dir:trace}\progidxonly{trace}\\
If you want to trace all predicates, use\\
\ind\prog{:-~trace~all.}\\
These directives are cumulative.


