\Chapter{Introduction\label{sec:intro}}

This document is a user manual for \about{dra}, an interpreter for tabled
logic programming with coinduction.

The interpreter implements ``top-down tabled programming'' via so called
``Dynamic Reordering of Alternatives'' \cite{guo-gupta-dra}.  It also
supports ``co-logic programming'', i.e., logic programs that contain
coinductive predicates \cite{coinductive}, \cite{co-LP}.

Appart from support for coinduction, there are two significant changes with
respect to the original description \cite{guo-gupta-dra}:
\begin{enumerate}

\item A tabled goal will never produce the same answer twice.

  More specifically: two answers will never be variants of each
  other.\footnote{
  Please note that in this document \about{goal} means an instance of a
  procedure call.}

\item By default, new answers for a tabled goal will be produced before old
  answers.  The user can reverse the order by means of a directive
  (\prog{:-~old\_first.}).

  A ``new answer for a tabled goal'' is an answer that has not yet been seen
  (and tabled) for a variant of the goal.

  The default behaviour is intended to help computations converge more
  quickly.  The user is given an option to change it, because some predicates
  may produce a very large (even infinite) set of answers on backtracking,
  and the application might not require those answers.
\end{enumerate}
