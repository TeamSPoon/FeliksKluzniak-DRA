\documentclass[12pt,draft]{report}
\usepackage{fancyvrb}
\usepackage{float}
\usepackage{varioref}
\usepackage[nottoc]{tocbibind} % Doesn't play well with package "index" :-(
\usepackage{makeidx,showidx}
\usepackage{index}  % prevents expansion in index entries, needed for progidx

%\proofmodetrue      %  <<<< (index) suppress for the final copy <<<<<<

\newindex{xpreds}{pdx}{pnd}{Index of procedures}


\makeindex

\setlength{\parindent}{2ex}
\setlength{\parskip}{0.5ex}

\floatplacement{listing}{htbp}

%----------------------------------------------------------------------
%  MACROS

% Facilitate transition between article and book:
\newcommand*{\Chapter}[1]{\chapter{#1}}
\newcommand*{\Section}[1]{\section{#1}}
\newcommand*{\Subsection}[1]{\subsection{#1}}
%\newcommand*{\Chapter}[1]{\section{#1}}
%\newcommand*{\Section}[1]{\subsection{#1}}
%\newcommand*{\Subsection}[1]{\subsubsection{#1}}


% Definition of a new concept:
\newcommand*{\defconcept}[1]{\textbf{\emph{#1}}\index{#1|textsl}}
% Ditto if the word is different from the index entry:
\newcommand*{\defconcepti}[2]{\textbf{\emph{#1}}\index{#2|textsl}}

% Definition in the glossary
\newcommand*{\defgloss}[1]{#1\index{#1|textsl}}
% Ditto if the word is different from the index entry:
\newcommand*{\defglossi}[2]{#1\index{#2|textsl}}

% A short form of a tiny \marginpar:
\newcommand*{\mpar}[1]{\marginpar{\tiny#1}}

% A word in the text that should be indexed:
\newcommand*{\Index}[1]{#1\index{#1}}

% A word that is the object of discussion:
\newcommand*{\about}[1]{\emph{#1}}

% A small piece of a concrete program etc.:
\newcommand*{\prog}[1]{\texttt{#1}}

% A ``pattern'', e.g., a generic variable within a concrete call:
\newcommand*{\patt}[1]{\textsl{#1}}

% A reference to an entity in a program:
\newcommand*{\progfrag}[1]{\about{#1}}

% An example of a term in the text:
\newcommand*{\term}[1]{\progfrag{#1}}

% A predicate specification:
\newcommand*{\pred}[1]{\about{#1}}

% An index entry for a procedure:
\newcommand*{\progidx}[1]{\index[xpreds]{#1@\prog{#1}}}

% A reference to a figure:
\newcommand*{\figref}[1]{Fig.~\vref{#1}}

% A reference to a listing:
\newcommand*{\lstref}[1]{Listing~\vref{#1}}

% A reference to a chapter:
\newcommand*{\chapref}[1]{chapter~\ref{#1}}

% A reference to a section:
\newcommand*{\secref}[1]{Sec.~\ref{#1}}

% An indentation
\newcommand*{\ind}{\hbox{\hspace{2em}}}

% Mathcal in normal text:
\newcommand*{\normcal}[1]{\(\mathcal{#1}\)}

% An explanation in Dijsktra-style proof via eqnarray*:
\newcommand*{\expl}[1]{\mbox{[~\emph{#1}~]}}

% Abbreviations for use in tabular:
\newcommand*{\h}{\hspace{1pt}}
\newcommand*{\mc}[3]{\multicolumn{#1}{#2}{#3}}

% The set of natural numbers: must be in math mode!
\newcommand*{\N}{\mathcal{N}}

% The powerset of something: must be in math mode!
\newcommand*{\Power}{\mathcal{P}}

% A binary relation between two arguments: must be in math mode!
% \binrel{a}{R}{c} to get a R b.
\newcommand*{\binrel}[3]{#1\,#2\,#3}

% The implication symbols: must be in math mode!
\newcommand*{\Implies}{\Rightarrow}
\newcommand*{\Implied}{\Lefttarrow}
\newcommand*{\IFF}{\Leftrightarrow}

% The disjunction symbol: must be in math mode!
\newcommand*{\Or}{\vee}

% The conjunction symbol: must be in math mode!
\newcommand*{\And}{\wedge}

% The dot that divides the quantifier and variables and the quantified
% formula: must be in math mode!
\newcommand*{\qdot}{ \;.\;}

% A universally quantified formula: must be in math mode!
\newcommand*{\All}[2]{(\forall #1 \qdot #2)}

% An equivalence class with for this relation, with this representative: must
% be in math mode!
\newcommand*{\EqClass}[2]{[#2]_{#1}}



%----------------------------------------------------------------------
%  ENVIRONMENTS

\floatstyle{boxed}
\newfloat{listing}{htbp}{lst}[chapter]
\floatname{listing}{Listing}

\newenvironment{warning}%
{\begin{quote}\textbf{Warning:}\itshape}%
{\end{quote}}


%----------------------------------------------------------------------
\title{The DRA Interpreter\\
User Manual}

\author{Feliks Klu{\'z}niak\\
  \emph{Applied Logic, Programming Languages and Systems Lab}\\
  \emph{Department of Computer Science}\\
  \emph{University of Texas at Dallas}
}
\date{\today}

\bibliographystyle{plain}


%----------------------------------------------------------------------
\begin{document}

\maketitle

\tableofcontents

\Chapter{Introduction\label{sec:intro}}

\Section{Overview\label{sec:overview}}

This document is a user manual for \about{dra}, an interpreter for tabled
logic programming with coinduction.  The manual has been written for users of
various versions of Unix: if you are running another system, various small
details (such as the way to enter the end-of-file character) may be
different.

The interpreter implements ``top-down tabled programming'' via so called
``Dynamic Reordering of Alternatives'' \cite{guo-gupta-dra}.  It also
supports ``co-logic programming'', i.e., logic programs that contain
coinductive predicates \cite{coinductive}, \cite{co-LP}.


Apart from support for coinduction, there are two significant changes with
respect to the original description \cite{guo-gupta-dra}:
\begin{Enumerate}

\item
  A tabled goal will never produce the same result twice.\footnote{
  In this document \Defconcept{goal} means an instance of a procedure call
  (i.e., an invocation of a predicate).  A \Defconcept{tabled
    goal}\index{goal!tabled} is a goal that invokes a tabled predicate. A
  \Defconcept{result} is the instantiation of a goal when it succeeds.}

  More precisely: a tabled goal will not succeed twice with instantiations
  that are variants of each other.\index{duplicate
    results}\index{result!duplicate}

\item
  By default, new results for a tabled goal will be produced before old
  answers.\index{order of results}\index{result!order}
  The user can reverse the order by issuing the directive\\
  \ind\prog{:-~old\_first.}\label{dir:old-first}\progidxonly{old\_first}

  A ``new result for a tabled goal''\label{new-result}\index{new
    result}\index{result!new} is a result that has not yet been tabled for
  this goal.  (More precisely: a result such that the table does not yet
  contain a variant of this result associated with a variant of this goal.)

  The default behaviour is intended to help computations converge more
  quickly.  The user is given an option to change it, because some predicates
  may produce a very large (even infinite) set of answers on backtracking,
  and the application might not require those answers.
\end{Enumerate}


%-------------------------------------------------------------------------------
\Section{Tabling\label{sec:tabling}}%
\index{tabled predicate}%
\index{predicate!tabled}

By default, each predicate that is defined in the interpreted program is a
``normal'' Prolog predicate.  If you want a predicate (\pred{p/2}, say) to be
tabled, you must put an appropriate declaring directive in your
program:%
\footnote{
    In literature about logic programming with tabling one often
    encounters the form \prog{:- table p/2}.  Since the directive is
    almost universally called a declaration (and not a command), the word
    \about{tabled} is obviously more appropriate.}\\
\ind\prog{:-~tabled~p/2.}\label{dir:tabled}\progidxonly{tabled}

When you declare a predicate as tabled, there will be
two consequences:
\begin{Enumerate}
\item
  Every time a goal with this predicate symbol (i.e., an invocation of this
  predicate) succeeds, the result (i.e., the resulting instantiation of the
  goal) is stored in a special table called the \Defconcept{answer
    table}\index{table|see{answer table}}.  Such a tabled result is commonly
  referred to as an \Defconcept{answer}.\index{answer}

  Once all the possible results have been so stored, subsequent invocations
  can draw upon the tabled answers, instead of recomputing them all over
  again. This is all done in a way that does not affect the semantics of the
  program, except perhaps for the order in which results are reported and the
  number of repetitions of the same result. (Of course, if your program is
  not sufficiently close to pure Prolog, e.g., if it relies on and/or
  produces side-effects, then all bets are off.)

\item
  When a goal that invokes a tabled predicate is found to be a variant of one
  of its ancestors on the recursion stack, the goal is not
  expanded.\footnote{
    Instead, it succeeds with results that have already been put into the
    table (or fails if there are none).  Things are arranged so that no
    results are lost.}
  This allows you to write your predicates in a more declarative fashion,
  without worrying, e.g., about the danger of left recursion.
\end{Enumerate}


Tabling is not always as advantageous as one might think, for two reasons:
  \begin{Enumerate}
  \item
    If the number of results is large, and goals are seldom repeated, the
    main effect might be a significantly increased demand for memory, which
    is not offset by a shorter processing time.

  \item It so happens that in order to preserve the semantics of the program
    the tabled answers must be associated with the goal that produced them,
    and are accessible only to variant goals.  Other goals that invoke the
    same predicate will require recomputation, and the computed results will
    increase the size of the table, even if they are the same as those that
    have already been stored.
  \end{Enumerate}


A more extensive discussion of tabling is well outside the scope of this user
manual.



%-------------------------------------------------------------------------------
\Section{Coinduction\label{sec:coindution}}%
\index{coinductive predicate}%
\index{predicate!coinductive}

The interpreter supports ``old style'' coinduction (see \cite{coinductive}
and \cite{co-LP}), as well as ``new style'' coinduction
\cite{gupta-coinductive-personal}.

If you want a predicate (\pred{p/2}, say) to be treated as an ``old style''
coinductive predicate, you must put the following declaring directive in your
program:\\
\ind\prog{:-~coinductive~p/2.}%
\label{dir:coinductive}\progidxonly{coinductive}

Such a declaration has one effect:
\begin{Itemize}
  \item
    If a goal that invokes a coinductive predicate is found to be unifiable
    with an ancestor on the recursion stack, the goal is unified with the
    ancestor and succeeds.  Upon backtracking it is unified---one by
    one\footnote{
      In \about{dra} this is done in reverse chronological order, i.e.,
      the more immediate ancestors are used first.
    }---with other such ancestors;
    it is expanded in the normal way (i.e., by
    using clauses) only when all the unifiable ancestors have been taken
    advantage of in this manner.
\end{Itemize}

If you want \pred{p/2} to be treated as a ``new style'' coinductive predicate,
use the declaration:\\
\ind\prog{:-~coinductive1~p/2.}%
\label{dir:coinductive1}\progidxonly{coinductive1}\\
This causes the predicate to be treated as above, except that clauses are used
only when there are no unifiable ancestors.\footnote{%
  The mnemonic value of our particular convention is that the ``new style''
  coinductive predicates give you only one way to satisfy a goal.}

The declaration \prog{:-~coinductive~p/2.} subsumes \prog{:-~coinductive1~p/2.}
if both are present in the same program.  If a predicate is declared as both
coinductive in the new style and tabled, then tabled answers are used even for
goals that have unifiable ancestors.

Declaring a predicate as coinductive gives it a radically new meaning, which is
often the appropriate one for operations on ``infinite'' terms (represented by
cyclic terms).  Again, further discussion of the concept is outside the scope of
this document.

\Chapter{Starting the interpreter\label{chap:starting}}

The interpreter is written in Prolog.  It is distributed in source
form.\footnote{%
  Please see the ``README'' files in the distribution tree: they will help you
  find your way around.}

The interpreter is known to run on Eclipse 6.0 and Sicstus 4.0.  If you plan
to run programs that take advantage of coinductive programming, you will
probably do better with Sicstus, which has much better support for cyclic
terms.

For both these systems, the simplest way to proceed is:
\begin{enumerate}
\item start your logic programming system;
\item type in the following directive:\\
  \ind\prog{:-~[~'}\about{Path}\prog{/tabling/dra'~]}\\
  where \about{Path} is the path to the root of the distribution tree.
\end{enumerate}

It may well be that things have been installed differently on your site.
This might be because the interpreter has been modified to run with a
different Prolog system, or because an immediately-loadable version has been
made available in some standard directory. The person responsible for the
local installation of the interpreter will provide you with more details.


\bibliography{bibliography}
\printindex
\end{document}
